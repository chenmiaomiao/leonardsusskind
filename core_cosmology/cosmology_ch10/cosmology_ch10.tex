\documentclass{article}

\usepackage{arxiv}

% for mathbb{1}
\usepackage{dsfont}
\usepackage{bbold}

\usepackage[utf8]{inputenc} % allow utf-8 input
\usepackage[T1]{fontenc}    % use 8-bit T1 fonts
\usepackage{hyperref}       % hyperlinks
\usepackage{url}            % simple URL typesetting
\usepackage{booktabs}       % professional-quality tables
\usepackage{amsfonts}       % blackboard math symbols
\usepackage{nicefrac}       % compact symbols for 1/2, etc.
\usepackage{microtype}      % microtypography
\usepackage{lipsum}
\usepackage{graphicx}
\graphicspath{ {./images/} }
\graphicspath{ {./media/} }
\usepackage{caption}
\usepackage{subcaption}

% for argmax
\usepackage{amsmath}
\DeclareMathOperator*{\argmax}{arg\,max}
\DeclareMathOperator*{\argmin}{arg\,min}
\DeclareMathOperator{\sign}{sgn}

\usepackage{amsthm}

\usepackage[bottom]{footmisc}

% for hyphen in matrix
\makeatletter
\newcommand{\longdash}[1][2em]{%
  \makebox[#1]{$\m@th\smash-\mkern-7mu\cleaders\hbox{$\mkern-2mu\smash-\mkern-2mu$}\hfill\mkern-7mu\smash-$}}
\makeatother
\newcommand{\omitskip}{\kern-\arraycolsep}
\newcommand{\llongdash}[1][2em]{\longdash[#1]\omitskip}
\newcommand{\rlongdash}[1][2em]{\omitskip\longdash[#1]}

% round symbol
\usepackage{mathtools, nccmath}
\DeclarePairedDelimiter{\nint}\lfloor\rceil
\DeclarePairedDelimiter{\abs}\lvert\rvert

% for quantum mechanics
\usepackage{braket}

% chinese
%\usepackage[utf8]{ctex}

% \usepackage{biblatex}
% \addbibresource{mlqm.txt}

\usepackage{cleveref}

% \usepackage{mathtools}
\DeclarePairedDelimiter\bbra{\langle}{\rvert}
\DeclarePairedDelimiter\kket{\lvert}{\rangle}
\DeclarePairedDelimiterX\bbraket[2]{\langle}{\rangle}{#1 \delimsize\vert #2}

\usepackage{scalerel,stackengine}
\stackMath
\newcommand\wwidehat[1]{%
\savestack{\tmpbox}{\stretchto{%
  \scaleto{%
    \scalerel*[\widthof{\ensuremath{#1}}]{\kern-.6pt\bigwedge\kern-.6pt}%
    {\rule[-\textheight/2]{1ex}{\textheight}}%WIDTH-LIMITED BIG WEDGE
  }{\textheight}% 
}{0.5ex}}%
\stackon[1pt]{#1}{\tmpbox}%
}
\parskip 1ex

\usepackage{mathrsfs}

\usepackage{diagbox}

\usepackage{makecell}

\usepackage{longtable}

\usepackage[acronym]{glossaries}

\usepackage{fancyvrb}

\usepackage{listings}

\usepackage{changepage}

\usepackage{authblk}

\usepackage{setspace}

\newcommand{\ignore}[1]{}

\newcommand{\beginsupplement}{%
        \setcounter{table}{0}
        \renewcommand{\thetable}{S\arabic{table}}%
        \setcounter{figure}{0}
        \renewcommand{\thefigure}{S\arabic{figure}}%
     }

\makeglossaries

% \newglossaryentry{duck}{name=duck,
%   description={a waterbird with webbed feet}}
% \newglossaryentry{parrot}{name=parrot,
%   description={mainly tropical bird with bright plumage}}

\setacronymstyle{long-short}

% \newacronym{rbm}{RBM}{ristricted boltzmann machine}
\newacronym{gnn}{GNN}{graph neural network}


% \title{Evolving systems and flowing entropy: Finding order in galaxies for cancer treatment, poverty reduction and astrosociology}
\title{Evolving systems and flowing entropy: Fit the lifetime of galaxies for cancer cure, poverty reduction and astrosociology with graph}

\author{
  Rongzhou Chen \\
  Precision Medicine and Public Health Department \\
  Tsinghua-Berkely Shenzhen Institute, Tsinghua University \\
  Shenzhen, China \\
  \texttt{crz18@mails.tsinghua.edu.cn} \\
}

% % \author{}
% \author[1,2]{Chen Miao}
% \author[1,2,3,4]{Shaohua Ma}
% \affil[1]{Tsinghua-Berkeley Shenzhen Institute, Tsinghua University, Shenzhen, 518055, China}
% \affil[2]{Tsinghua University Shenzhen International Graduate School, Shenzhen, 518055, China}
% \affil[3]{Corresponding email: ma.shaohua@sz.tsinghua.edu.cn}
% \affil[4]{Lead Contact}
% \date{}                     %% if you don't need date to appear
% \setcounter{Maxaffil}{0}
% \renewcommand\Affilfont{\itshape\small}

\begin{document}
% \doublespace 

\maketitle
%\begin{abstract}
%
%\end{abstract}


\keywords{Nebula \and Galaxy \and Cancer \and Poverty \and Astrosociology}


%%%%%%%%%%\\\\\\\\\\%%%%%%%%%%\\\\\\\\\\%%%%%%%%%%\\\\\\\\\\%%%%%%%%%%\\\\\\\\\\%%%%%%%%%%\\\\\\\\\\%%%%%%%%%%\\\\\\\\\\
%%%%%%%%%%//////////%%%%%%%%%%//////////%%%%%%%%%%//////////%%%%%%%%%%//////////%%%%%%%%%%//////////%%%%%%%%%%//////////

\glsresetall

\section{Background}
  \subsection{Time, duration and evolving systems from galaxies to artificial intelligence(AI)}
  % With a simple thought experiment, I always imagine that everything in the universe is fixed, including every celestial body, atom, subatomic particle. There is no more evolution then as well as time. 

  What's time if everything in this universe is fixed? Is It change of sructure that creates time? Or it's time gives structure the 'space' to change?
  
  % \subsection{Treating cancer with `nationalizing' but not killing}
  % When I heard the word `nationalizing' at first time, it is related to football player in Japan. The Japanese government encourages Japanese women to marry with men from the  race prevailing in football. Gradually, more nationalized Japanese will become dominant in football playing instead of foreign aid.
  
  % This is also happened in galaxies. One galaxy might engulf its neighbors.
  
  % In current medicine, especially in tumor treatment, killing tumor cells is in the first priority. Nevertheless, as we all know, that simply doesn't work.
  
  \subsection{Entropy gradient, poverty reduction and thinking 1000 years in advance for civilization}
  % gradient, Laplacian, 2nd drivative of entropy
  
  Decreasing entropy of a local region can increase the entropy gradient in a global view. 
  
  % Elon Musk
  
  
  \subsection{Graph neural networks (GNNs) and quantum computing}  
  Machine learning embraced its booming era. On the contrast, quantum computing is still in its infancy. As an emerging area, quantum machine learning attracted much attention from researchers.

%%%%%%%%%%\\\\\\\\\\%%%%%%%%%%\\\\\\\\\\%%%%%%%%%%\\\\\\\\\\%%%%%%%%%%\\\\\\\\\\%%%%%%%%%%\\\\\\\\\\%%%%%%%%%%\\\\\\\\\\
%%%%%%%%%%//////////%%%%%%%%%%//////////%%%%%%%%%%//////////%%%%%%%%%%//////////%%%%%%%%%%//////////%%%%%%%%%%//////////


\section{Objectives}
  % focusing on a topic
  
  Stability might be the most unstable property of a system. Changes happened internally and externally transit a system from one state to the another.
  
  % \subsection{Back to square one: Why can astrophysics be the answer for all?}

\section{Proposed Method}
  \label{proposed_method}

  % \subsection{Identifying the principle of stably evolving celestial system with graph model}
  \subsection{Regress lifetime celestial bodies and systems with graph model}





    

%%%%%%%%%%\\\\\\\\\\%%%%%%%%%%\\\\\\\\\\%%%%%%%%%%\\\\\\\\\\%%%%%%%%%%\\\\\\\\\\%%%%%%%%%%\\\\\\\\\\%%%%%%%%%%\\\\\\\\\\
%%%%%%%%%%//////////%%%%%%%%%%//////////%%%%%%%%%%//////////%%%%%%%%%%//////////%%%%%%%%%%//////////%%%%%%%%%%//////////

% \section{Acknowledgements}


% If you publish results when using this database, then please include this information in your acknowledgements.

% \nocite{*}

% \printbibliography
\bibliographystyle{unsrt}  
\bibliography{ref}  %%% Remove comment to use the external .bib file (using bibtex).

%%%%%%%%%%\\\\\\\\\\%%%%%%%%%%\\\\\\\\\\%%%%%%%%%%\\\\\\\\\\%%%%%%%%%%\\\\\\\\\\%%%%%%%%%%\\\\\\\\\\%%%%%%%%%%\\\\\\\\\\
%%%%%%%%%%//////////%%%%%%%%%%//////////%%%%%%%%%%//////////%%%%%%%%%%//////////%%%%%%%%%%//////////%%%%%%%%%%//////////
%%%%%%%%%%\\\\\\\\\\%%%%%%%%%%\\\\\\\\\\%%%%%%%%%%\\\\\\\\\\%%%%%%%%%%\\\\\\\\\\%%%%%%%%%%\\\\\\\\\\%%%%%%%%%%\\\\\\\\\\
%%%%%%%%%%//////////%%%%%%%%%%//////////%%%%%%%%%%//////////%%%%%%%%%%//////////%%%%%%%%%%//////////%%%%%%%%%%//////////
%%%%%%%%%%\\\\\\\\\\%%%%%%%%%%\\\\\\\\\\%%%%%%%%%%\\\\\\\\\\%%%%%%%%%%\\\\\\\\\\%%%%%%%%%%\\\\\\\\\\%%%%%%%%%%\\\\\\\\\\
%%%%%%%%%%//////////%%%%%%%%%%//////////%%%%%%%%%%//////////%%%%%%%%%%//////////%%%%%%%%%%//////////%%%%%%%%%%//////////

\newpage

% \appendix
% \beginsupplement

\begin{center}
% \textbf{\large Supplemental Materials: Power transformation of quantum state}
% A Quantum Theory Incorporated Machine Learning Technique to Find Linearly Maximum Separable Components
\end{center}

\setcounter{section}{0} 
\setcounter{equation}{0}
\setcounter{figure}{0}
\setcounter{table}{0}
\setcounter{page}{1}

\makeatletter

\renewcommand{\thesection}{S\arabic{section}}
\renewcommand{\theequation}{S\arabic{equation}}
\renewcommand{\thefigure}{S\arabic{figure}}
\renewcommand{\thetable}{S\arabic{table}}
% \renewcommand{\bibnumfmt}[1]{[S#1]}
% \renewcommand{\citenumfont}[1]{S#1}}



\printglossaries

\end{document}